\section{Other Proof-of-Stake Systems}\label{sec.other}

We have presented our construction in the Ouroboros proof-of-stake model. Our construction
is quite general and can be adopted to any proof-of-stake system. Many proof-of-stake systems are split
into (potentially smaller) epochs in which some sampling from the underlying stake distribution
is performed according to some random number. The random number generation can be performed in multiple
ways. For example, all of Ouroboros, Ouroboros Praos, and Ouroboros Genesis use a verifiable secret
sharing mechanism, while
Algorand uses a multiparty computation. The stake distribution from which the sampling is performed
could also have various nuances such as delegation, might require locking one's funds, may exclude
people with very small stake, or may give different weights to different stake ownership. In all
of these cases, a frozen stake distribution from which the final sampling is performed is determined.

Our scheme can be generalized to any proof-of-stake scheme in which the leader can be verified
from a frozen stake distribution and some randomness, no matter how it is generated, as
long as the block associated with a particular slot can be uniquely determined after it stabilizes
(a property that follows in any blockchain system as long as it observes the common prefix property).
In the scheme we described throughout the paper, the sequence of signatures $\overline{\sigma}_{j+1}$ that are generated
in an epoch $j$ and vouch for the leaders of the next epoch sign the public key sequence $\overline{pk}_{j+1}$
of the next epoch. To generalize our scheme to any proof-of-stake system with randomness and a stake distribution,
the signatures $\overline{\sigma}_{j+1}$ need not sign the public key sequence any more; instead, they
can sign:

\begin{enumerate}
  \item the epoch randomness $\eta_{j+1}$ of the next epoch, and
  \item the frozen stake distribution $\textsf{SD}_j$ of the current epoch
        that will be used for sampling during the next epoch.
\end{enumerate}

Of course, in
such a scheme, a succinctness problem arises: The stake distribution $\textsf{SD}_j$ might be large.
However, this problem can be overcome by organizing the stake distribution $\textsf{SD}_j$ into a
Merkle tree. This Merkle tree contains one leaf for every satoshi (the smallest cryptocurrency
denomination). The leaf's value is the public key who owns this satoshi. When the sampling from
$\textsf{SD}_j$ according to the randomness $\eta_{j+1}$ occurs, the prover can provide a proof
that the correct leader was the one that happened to be elected by opening the particular Merkle
tree path at a particular index. That way, the verifier can deduce the $2k$ last slot leaders of each epoch. Because
the number of satoshis can be large, this Merkle tree can have a large (potentially an exponential)
number of leaves.
However, its root and proofs can be efficiently computed using sparse Merkle tree techniques~\cite{sparse-mt}
(or Merkle tries~\cite{wood})
because the tree contains a polynomial number of continuous ranges in which many consecutive
leaves share the same value.
Even better, a Merkle--Segment trees~\cite{flyclient}) can be used. These trees are similar to
Merkle trees, except that each node (internal or leaf) is also annotated with a numerical value,
here the total stake under the subtree rooted at the particular node. Each internal node has
the property that its annotated value is the sum of the annotated values of its children.

The above technique is quite generic, but each system has its nuances that must be accounted for.

\noindent
\textbf{Ouroboros Praos and Genesis.}
These two protocols have some similarities to Ouroboros, but also significant differences.
As Ouroboros Praos and Genesis are designed to be resilient to fully
adaptive adversaries, the actual slot leader of each slot is not known \emph{a priori}.
However, a party can himself determine whether he is eligible to be the
slot leader by evaluating a VRF on the epoch randomness and the current slot index
using his private key.
If the VRF output is below a certain threshold, determined by the candidate leader's
stake, then the party is elegible to be a leader for this slot.
The party's public key can then be used by others to verify a proof that the VRF computation
was correct, and that he is indeed a rightful leader.

Because we cannot determine the leaders of the $j+1^\text{st}$ epoch at the end of epoch
$j$, we cannot hope to have the leaders of the $j^\text{th}$ epoch sign off the public
keys of the leaders of epoch $j+1$. However, the above technique,
in which signatures sign the randomness
and a Merkle--Segment tree of the stake distribution, together with the VRF proof, suffices.
In this construction, the signatures of epoch $j$ sign off the randomness for epoch $j+1$ and stake
distribution Merkle tree for epoch $j$. At a later time, when it is revealed who is leader,
the honest prover can provide the VRF proof to the verifier, and the verifier can check
that the leader was indeed rightful. To obtain the VRF threshold, the prover can open the
Merkle--Segment tree to the depth required to illustrate the total sum of the stake of the leader.
Once the leader's stake is revealed, the threshold used in the VRF inequality is validated.

Because each slot can have multiple leaders,
if approached na\"ively, this can cause the handover trees
to lose their deterministic property. Instead, the \emph{handover witness} technique must be
used to illustrate that the provided leaders actually ended up being part of the chain, similarly to
how we solved the problem in the case of Ouroboros in the main body of the paper.

These protocols have several advantages, including security in the semi-synchronous
setting as well as resilience to adaptive adversaries~\cite{praos,genesis}.

\noindent
\textbf{Snow White.}
This protocol uses epochs and every epoch contains a randomness and a
stake distribution from which leaders are sampled~\cite{snowwhite}.
Therefore, our protocol can be readily
adapted to it.

\noindent
\textbf{Harmony.}
The Harmony chain also splits the execution into epochs of duration of roughly one
day~\cite{harmony}.
There is a stake distribution from which sampling is done, although this distribution
is slightly altered for the sampling purposes as compared to money ownership purposes.
As such, our protocol would need to use the stake distribution used for sampling.
The protocol already provisions for \emph{epoch blocks}, a representative larger block
in each epoch that contains a sample of leaders such that any sufficiently large subset
of them will contain one honest party. The leaders recorded in epoch blocks can be used
for the purposes of our protocol. For the handover transitions, the leaders recorded in
the previous epoch block are expected to sign off the next epoch block. Harmony already
leverages this mechanism to conduct cross-chain transfers in a protocol that has
linear efficiency. Our protocol can be readily applied on top to achieve our described
exponential improvements.

\noindent
\textbf{Algorand.}
Contrary to Ouroboros, Algorand offers immediate finality~\cite{algorand}.
Once a block is broadcast,
any transactions contains within are confirmed and can no longer be reverted. In other
words, its common prefix property holds with a parameter of $k = 1$. To achieve this,
Algorand runs a full Byzantine Agreement protocol for the generation of every block
before moving to the next block. One way to look at it is to think of Algorand as a
coin in which the epoch duration is $R = 1$. Our construction can therefore create a
handover tree in which the leaves are exactly the blocks in the Algorand chain. The
Algorand private sortition mechanism can be used to elect a committee large enough
to ensure honest supermajority (a property required for Algorand's security).
This committee, whose members can be placed in increasing order by their public
key to ensure determinism, can then be used in place of our sequence of public keys,
to sign off the results of the next block. Even though our handover
tree now becomes slightly larger, with its number of leaves equal to the
chain length $|\chain|$, our protocol is still $\mathcal{O}(\log|\chain|)$.
