\section{The PoPoS Primitive}

We begin by describing the goal of our protocol.
Inspired by the analogous proof-of-work constructions~\cite{nipopows},
our model consists of an honest party $V$,
the verifier, holding only the Genesis block $\mathcal{G}$, who connects to multiple
\emph{prover} parties $\mathcal{P} = \{P_1, P_2, \cdots, P_q\}$, at least one of which is honest
(the so-called \emph{any-trust}~\cite{arbitrum} or \emph{non-eclipsing}~\cite{eclipse,eclipse-ethereum}
assumption). All provers beyond the one honest can be controlled by the adversary,
and the verifier does not know which party among the provers is the honest one.
The honest prover parties, beyond acting as provers for the verifier, are online full nodes participating
in the proof-of-stake system, and each of them maintains a chain $\chain^{P_i}$.
The job of the verifier is to discern the honest from the adversarial provers in order to extract
information about the underlying chain. More specifically, the verifier wishes to
discover the longest chain among a sea of shorter or invalid chains, and to deduce
a fact about it, such as whether a particular payment was made.

The protocols $V$, ran by the honest verifier, and $P$, ran by the honest prover, are
\emph{multi-round interactive} protocols. When $V$ connects to the set $\mathcal{P}$ of provers,
it can interrogate each of them, and may require the help of the other provers in this process.
After the interrogation is over, the verifier outputs the \emph{current application state} of
the underlying proof-of-stake blockchain. The application state is the commitment made by
the blockchain to the current application data of the system: Either the current \emph{UTXO}
tree, in the UTXO model~\cite{sok}, or the current \emph{accounts} tree, in the accounts
model~\cite{buterin} (we assume that each block contains a commitment to the current
application state~\cite{utreexo}). It is not necessary that the verifier outputs the
\emph{whole} application state---a short Merkle tree commitment to it is sufficient.
Subsequently, facts about the current state, such as whether a payment has been made,
can be proven using standard Merkle tree proof techniques. The application state is
sufficient for all the purposes of a wallet: Receiving and validating incoming payments
and making new payments.

We wish for the verifier to output a state consistent with the view of the honest parties.
Ideally, this would be a state committed in the most recent stable block $\chain[-k]$ of
an honestly adopted chain $\chain$.
The transcript of all messages exchanged between $V$ and $\mathcal{P}$ after a protocol
execution is completed is denoted $\textsc{VIEW}_{V \Leftrightarrow \mathcal{P}}$
(the transcript contains only messages that have been processed by the receiver).

\begin{definition}[Proof of Proof-of-Stake]
  A \emph{Proof of Proof-of-Stake protocol} (PoPoS) is a pair of interactive probabilistic polynomial-time
  algorithms $(P, V)$. The algorithm $P$ is the \emph{honest prover} and the algorithn $V$
  is the \emph{honest verifier}. The algorithm $P$ is ran by an online participating full node
  in a proof-of-stake system, while $V$ is ran by a node booting up for the first time
  holding only the genesis block $\mathcal{G}$. The protocol is executed between $V$ and a
  set of provers $\mathcal{P}$.
\end{definition}

Our security definition concerns the case of an honest verifier communicating with
a set of possibly malicious provers (the case of a malicious verifier is of no concern).

\begin{definition}[Security]
  When executed with a set of provers $\mathcal{P}$ at least one of which is honest,
  the verifier $V$ outputs the current application state of the chain in a manner
  \emph{consistent} with the view of the honest parties: It either outputs a state
  adopted by an honest party, or a state which \emph{will be adopted} by an honest party
  within $k$ blocks.
\end{definition}

A trivial Proof of Proof-of-Stake protocol is to have $V$ run the full node code. In
that case, the transcripts $\textsc{VIEW}_{V \Leftrightarrow \mathcal{P}}$ are long.
The challenge we will take up is to make this protocol more efficient, or \emph{succinct}.
Our goal is to design sublinear protocols.

\begin{definition}[Succinctness]
  Consider a PoPoS protocol $(P, V)$ with a set of provers $\mathcal{P}$ operating in
  an underlying proof-of-stake system where the longest honestly adopted chain is $\chain$.
  The protocol is \emph{succinct} if every transcript length
  $|\textsf{VIEW}_{V \Leftrightarrow \mathcal{P}}| \in \mathcal{O}(\textsf{poly}\log|\chain|)$
\end{definition}

In the above definition, we assume that $\mathcal{P}$ is of constant size.
