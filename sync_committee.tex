\subsection{Sync Committee \& Light Client}
\label{sec:sync-committee}
  
The flagship feature of the recent Altair hard fork is a minimal light client construction
for the PoS Ethereum beacon chain \cite{minimal_light_client}.
The construction adds two components to the specification: \emph{Sync committees} and the functionality for 
the light clients to find an up-to-data canonical beacon chain header with the help of the sync committees.
  
\paragraph{Sync Committee} 
Sync committee consists of 512 validators that are sampled uniformly at random from the validator 
set every sync committee \emph{period}.
Each period lasts $256$ epochs, approximately $27$ hours.
During the period, each validator of the corresponding committee is responsible for signing the header of the block
viewed as the new tip of the beacon chain at each slot.
% Can we change this to signing each confirmed block?
The proposer of each slot in turn collects the sync committee signatures on the hash of the block 
that was observed as the tip of the beacon chain in the previous slot, 
and includes the aggregated signature in the header of its proposal as part of a $\mathsf{SyncAggregate}$ struct
%(\cf \href{https://github.com/ethereum/consensus-specs/blob/dev/specs/altair/beacon-chain.md#beaconblockbody}{BeaconBlockBody}) 
\cite{altair}.
The $\mathsf{SyncAggregate}$ struct also contains a bit vector of size 512 that indicates the sync committee members 
who have contributed to the aggregate signature.
Each sync committee member that contributed to the signature is rewarded a fixed amount, 
whereas the remaining members are penalized by the same amount.   
   
\paragraph{Light Client}

Light client relies on the sync committee signatures to detect the last beacon blocks finalized by the consensus protocol.
%Are we assuming that the light client tries to boot up with the Casper FFG finalized blocks? 
In this context, to identify the last finalized block, light client has to only verify the sync committee for each period instead 
of verifying the beacon blocks themselves or using the stake distribution table.
To distinguish the block identified by the light client with the help of the sync committee from those finalized by the consensus
protocol, we will call the earlier ones \emph{sync-finalized}, and the latter \emph{consensus-finalized}.
An ideal light client protocol ensures that the latest sync-finalized block in the view of a light client is a recent
consensus-finalized block, a property captured by the light client security (\cf Definition~\ref{def:light-client-security}$

The full view of a light client at a given round, summarized by a $\mathsf{LightClientStore}$ object, 
consists of the following core components:
\begin{itemize}
    \item the last sync-finalized beacon block header, $\mathsf{finalized\_header}$.
    \item the current sync committee, $\mathsf{current\_sync\_committee}$, corresponding to the period
    containing the slot of $\mathsf{finalized\_header}$.
    \item the next sync committee $\mathsf{next\_sync\_committee}$ for the next period. 
    As the state of each beacon block specifies the sync committee for the next period,
    $\mathsf{next\_sync\_committee}$ matches the committee specified by $\mathsf{finalized\_header}$. 
\end{itemize}

The light client updates its view when it receives a $\mathsf{LightClientUpdate}$ object over the network, 
which consists of the following parts: 
\begin{itemize}
    \item A beacon block header, $\mathsf{attested\_header}$, signed by the corresponding sync committee.
    \item A beacon block header, $\mathsf{finalized\_header}$, to be accepted as the new sync-finalized block.
    \item A new next sync committee $\mathsf{next\_sync\_committee}$ to replace the old one within the $\mathsf{LightClientStore}$.
    \item A Merkle inclusion proof $\mathsf{next\_sync\_committee\_branch}$ for the $\mathsf{next\_sync\_committee}$.
    \item A Merkle inclusion proof $\mathsf{finality\_branch}$ for the $\mathsf{finalized\_header}$.
    \item A $\mathsf{sync\_aggregate}$ object, consisting of an aggregate BLS signature and a bit vector that specifies
    the sync committee members that has contributed to the signature.
    \item A signature slot, $\mathsf{signature\_slot}$ that corresponds to the slot of the $\mathsf{attested\_header}$.
\end{itemize}

Upon receiving an update, the light client verifies it through the following process:
It first checks if $\mathsf{signature\_slot}$, the slot of the $\mathsf{attested\_header}$,
is larger than that of the $\mathsf{finalized\_header}$, and the slot of its last sync-finalized block.
It also verifies if $\mathsf{signature\_slot}$ is within the same or the next period given the period 
corresponding to the last sync-finalized block.
The client rejects updates containing signatures from sync committees that are more than one period in the future.

Next, the client verifies the inclusion of the $\mathsf{finalized\_header}$ within the state 
of the $\mathsf{attested\_header}$ through the $\mathsf{finality\_branch}$.
Then, if $\mathsf{finalized\_header}$ is from the next period, the client verifies the inclusion of the
$\mathsf{next\_sync\_committee}$ within the state of $\mathsf{finalized\_header}$ through $\mathsf{next\_sync\_committee\_branch}$.
Finally, it validates the aggregate signature within the $\mathsf{sync\_aggregate}$ on the $\mathsf{attested\_header}$ 
using the bitmap and the sync committee of the period containing the $\mathsf{signature\_slot}$.
The client knows the sync committee as the period containing the $\mathsf{signature\_slot}$ is either the same as 
that of the last sync-finalized block, or one larger, in which case it uses the $\mathsf{current\_sync\_committee}$ or 
the $\mathsf{next\_sync\_committee}$ within its $\mathsf{LightClientStore}$ respectively to verify the aggregate signature.

It is possible that an update does not specify a $\mathsf{finalized\_header}$. 
In this case, the light client treats the $\mathsf{attested\_header}$ as the header to be accepted 
as the new sync-finalized block, and goes through the same checks described above.

Finally, upon verifying an update, if the $\mathsf{attested\_header}$
was signed by over $2/3$ of the corresponding sync committee, then the light client replaces
its last sync-finalized beacon block header with the $\mathsf{finalized\_header}$ within the update if it
was specified, and with the $\mathsf{attested\_header}$ otherwise.
If the new sync-finalized block is from a higher period, then the client also updates 
the $\mathsf{current\_sync\_committee}$ and the $\mathsf{next\_sync\_committee}$ within its view.
In particular, the new $\mathsf{current\_sync\_committee}$ attains the value of the old $\mathsf{next\_sync\_committee}$ 
within the $\mathsf{LightClientStore}$, whereas the new $\mathsf{next\_sync\_committee}$ becomes 
the $\mathsf{next\_sync\_committee}$ contained in the update, \ie, the committee within the new sync-finalized block.

Due to network delays or temporary adversarial majorities, it might occasionally be the case that
no consensus-finalized beacon block within a period is signed by $2/3$ of the corresponding sync committee. 
Then, the light client updates cannot be verified, and the client cannot catch up with the tip of the consensus-finalized chain.
The current official Ethereum specification recommends using some heuristic methods to allow the client verify
the latest block headers.
However, despite succeeding in a forced update, such heuristics might harm the light client security
under certain adversarial conditions.
Thus, in this work, we assume that the underlying consensus protocol is not subject to such disruptions, 
and consider the single update mechanism described above.
%Should we also describe the forced update upon timeout? The regular update mechanism should be sufficient for our
% super light client mechanism.

A sequence of valid light client updates corresponding to old periods is sufficient for a light client 
to identify a finalized block from consecutive periods, up until the latest period.
This enables the light client to bootstrap with the consensus-finalized header chain by downloading only
25 kB per period: 24576 bytes for the 512 48-byte public keys in the sync committee, plus a few hundred bytes for auxiliary data.
However, since the total number of periods is linear in the time passed and the number of blocks in the consensus-finalized chain,
the light client would still be downloading data linear in the size of the header chain.  
To further reduce the download and execution complexity of a bootstrapping light client, we next describe how the
PoPoS protocol can be used in the context of Ethereum light clients.

\paragraph{Security for Light Clients}

Let $\L^{\cup}_{r'}$ denote the chain that corresponds to the \emph{union} of all consensus-finalized chains held by the honest validators at round $r'$.
Similarly, let $\L^{\cap}_{r}$ denote the chain that corresponds to the \emph{intersection} of all consensus-finalized chains held by the honest validators at round $r$.
Given these definitions, security for a light client protocol, \eg, Ethereum's sync protocol, can be expressed as follows:
\begin{definition}
\label{def:light-client-security} 
A light client protocol is secure with safety parameter $\nu$, if the sync-finalized beacon block header,
obtained by the client at the end of the protocol execution at round $t$ satisfies safety and liveness as defined below:

There exists a ledger $\L$ such that $\L[{:}-1]$ is the sync-finalized beacon block, and for all rounds $t' \geq t + \nu$:
\begin{itemize}
    \item \textbf{Safety:} $\L$ is a prefix of $\L^{\cup}_{r'}$.
    \item \textbf{Liveness:} $\L$ is a suffix of $\L^{\cap}_{r}$.
\end{itemize}

\end{definition}

  
   