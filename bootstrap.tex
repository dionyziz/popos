\section{Bootstrapping from Tree Roots}

\import{./}{algorithms/alg.bootstrap-prover}
\import{./}{algorithms/alg.bootstrap-verifier}

The handover trees developed in the previous section are immensely helpful in the superlight
client synchronization. Through the bisection game, we have already shown that the tree that
the honest verifier will adopt corresponds to the most epochs adopted by any honest party, or
one extra epoch that will soon by adopted by the honest parties. Now that the verifier has
obtain this tree and confirmed its veracity, the task that remains is to discern facts about
the underlying blockchain. We will show how the verifier can deduce whether a particular transaction
$\tx$ is confirmed or not. Similar questions about the blockchain such as how much money one
owns, or the state of a particular smart contract, can be resolved in a similar manner.
Our protocol will allow proving transaction inclusion in the stable part of the blockchain
(i.e., we can only speak about transactions that have been confirmed). More specifically,
can prove the inclusion of any transaction that any honest full node sees
as confirmed in their underlying chain. This includes transactions in the current epoch,
as well as older transactions since genesis. On the other hand, a transaction that no
honest party sees (or will soon see) as confirmed cannot be proven.

To bootstrap their current state, the verifier can ask all provers to provide him with
transactions of interest for his wallet. From the non-eclipsing assumption, all the confirmed
transactions of interest will be given to the verifier. In addition, the adversary may provide
transactions that are not of interest, or transactions of interest that have not been confirmed
and may never be confirmed. The verifier can distinguish between these by running our bootstrapping
protocol illustrated in Algorithm~\ref{alg.bootstrap-verifier} for each transaction. As long as the number
of transactions of interest is small (constant), the asymptotic performance will not be harmed. 
Practical performance can be improved by combining the querying of multiple transactions.

The algorithm works as follows: Once the correct root $\pi$ has been ascertained, the verifier
contacts each prover $P$ to verify whether the transaction $\tx$ is included in a chain established
during the epochs described by the tree whose root is $\pi$. The adversarial prover will not be
able to prove inclusion for an invalid transaction, but the honest prover will be.
The verifier makes a series of requests to the prover and receives some responses.
If the prover fails at any time to provide a valid
response to the verifier, the verifier deduces that the prover does not know how to prove inclusion
for the particular transaction, and continues with the rest of the provers.

Initially, the verifier requests the \emph{penultimate} leaf, of index $n-1$, in the tree $\pi$.
The leaf corresponds to the previous epoch from the current one.
The prover provides the leaf that contains $\overline{pk}^*_{j-1}$, the leaders of the
slots $S_{j-1}[-6k{:}-4k]$ of the previous epoch. In addition, the leaf contains a signature
$\sigma_{j-1}$ that we will not make use of in this stage. The prover also provides an inclusion
proof $z_1$ that shows that this is the correct leaf at index $n-1$ in $\pi$.
The prover then presents a subchain $\chain^*$ of at least $k+1$ connected blocks produced
during the slots $S_{j-1}[-6k{:}-4k]$. The verifier ensures that $\chain^*$ is a valid
connected chain of length at least $k+1$ and that its blocks have been created by the rightful
leaders as defined by $\overline{pk}^*_{j-1}$. The verifier then obtains $b^* = \chain^*[0]$, treating
it as if it were an honest block, and retrieves the root $b^*\textsf{.leaders}$ of the tree of leaders.
Lastly, the prover provides the $2k$ slots $S$ after the inclusion of $\tx$ in a block and the
public keys $\overline{pk}'$ of the leaders of these slots, together with an inclusion proof of
the leaves $\overline{pk}'$ at positions $S$ in the tree root $b^*\textsf{.leaders}$. Once the verifier
verifies the inclusion proof, he requests the particular block $b$ in which $\tx$ was confirmed.
He checks that $b$ is in $\chain^*$. Lastly, he requests an inclusion proof of $\tx$ in $b\textsf{.txs}$
which is verified according to the workings of a usual SPV client.

The following lemmas give the desired properties of the bootstrapping protocol.

\begin{lemma}[Bootstrapping Succinctness]
  Given a tree constructed correctly with the current number $n$ of epochs, or one extra epoch, the protocol
  of Algorithms~\ref{alg.bootstrap-prover}-\ref{alg.bootstrap-verifier}
  requires $\Theta(n)$ communication and requires $1$ round of interaction.
\end{lemma}
\begin{proof}
  \noindent
  \emph{Interactivity.}
  Each of the requests does not depend on the response
  processing on the verifier end. Therefore all interaction can be combined
  into one round and the protocol proceeds by making only a single request (containing $\tx$)
  from the verifier to the prover, and receiving a
  single response from the prover to the verifier. 

  \noindent
  \emph{Succinctness.}
  The data sent from the verifier to the prover is the transaction and the tree root of
  constant size. The prover responds with a collection $\overline{pk}_{n-1}$ of $6k$ keys
  (the respective signature is not processed at this stage), which is a constant. The
  chain chunks $\chain^*$ and $\chain'$ provided are each at most $2k$ blocks, also a constant.
  Lastly, the proofs $z_1$, $z_2$, and $z_3$ are $\Theta(\log n)$, $\Theta(\log n)$, and $\Theta(\log |b\textsf{.tx}|)$,
  the last of which is also a constant (the number of transactions in a single block).
  Overall, the communication complexity is $\Theta(\log n)$.
\end{proof}

\begin{lemma}[Bootstrapping Correctness]
  An honest prover $P$
  seeing a transaction $\tx$ as confirmed and
  executing the protocol of Algorithm~\ref{alg.bootstrap-prover}
  will convince the verifier holding an admissible handover tree root $\pi$
  about the inclusion of $\tx$ in $\pi$.
\end{lemma}
\begin{proof}
  Consider the adopted chain $\chain$ of the honest party $P$ and let $n$ be the number
  of epochs $P$ has observed and $m$ be the number of epochs in $\pi$.
  Since $\tx$ is confirmed, it must be included
  in a block $b$ of $\chain$ generated during some slot $sl$.
  Additionally, since $\tx$ is confirmed, the current slot must be at least $sl + 2k$.
  Since $\pi$ is admissible, $m = n$ or $m = n + 1$.
  In any case, $P$ has enough data to reconstruct the whole tree
  and certainly an inclusion proof $z_1$ of
  the penultimate leaf of this tree (if $m = n+1$,
  the one extra leaf required to construct the siblings in the inclusion proof
  will have been provided during the execution of Algorithm~\ref{alg.prover}).
  The penultimate leaf contains $\overline{pk}^*_{m-1}$ in agreement
  with the view of $P$ about the leaders in $S_{m-1}[-6k{:}-4k]$.
  Let $\chain^*, b^*$ be as in the algorithm.
  By the chain growth property, at least $k+1$ blocks of $\chain$
  must have been generated during $S_{m-1}[-6k{:}-4k]$.
  Therefore $|\chain^*| \geq k + 1$.
  During those slots, the randomness $\eta_m$ of epoch $m$
  and the stake distribution $\textsf{SD}_{m-1}$ used for the leader election
  of epoch $m$ are known and correctly included in $\chain^*$
  (otherwise they would not have been adopted by $P$).
  The leaders $b^*\textsf{.leaders}$ are for the same reason also correct
  and must contain all leaders up to and including epoch $m$, and
  in particular the leaders of the slots $S = \{sl, \cdots, sl + 2k\}$.
  Therefore, the prover has enough data to also construct the inclusion
  proof $z_2$. Let $\chain'$ be as in the algorithm. From chain growth,
  $|\chain'| \geq k + 1$. Lastly, $b$ will be in $\chain'$ by choice,
  and $\tx$ will be in $b$, therefore $P$ can also construct a valid $z_3$,
  as desired.

  There is one detail needed for the above algorithm to fully work: In case the
  admissible tree held by the verifier has $m$ epochs, but the honest prover
  has only seen $n = m - 1$ epochs, the prover must somehow know (1) the data
  in the last leaf $\overline{pk}_m$, as well as (2) the leaders of epoch $m$
  in order to construct a proof for $b\textsf{.leaders}$. Both of these pieces
  of data are available to the honest prover in case $n = m - 1$. To see this,
  observe that an admissible tree with $m = n + 1$ can only be constructed given
  a witness for the transition to epoch $n + 1$. This witness consists of at
  least $k+1$ blocks produced during the slots $S_n[-4k{:}-2k]$ and at least
  $k+1$ blocks produced during the slots $S_n[-2k{:}]$. By the honest
  subchain property, at least one block in the first range must be honestly
  generated (call it $b_1$), and at least one in the second range must be
  honestly generated too (call it $b_2$). Since $b_2$ is honestly generated
  at a slot after $b_1$, this means that the honestly generated block $b_1$
  was broadcast during its generated slot and has been received by $P$.
  This block $b_1$ contains both the leaders
  $b_1\textsf{.leaders}$ and of course the public keys $\overline{pk}_m$
  since they are a subset of the leaders. Therefore, even though $P$ may
  not yet have transitioned to epoch $m$, he certainly knows all the data
  required to provide the necessary proofs to $V$.
\end{proof}

\begin{lemma}[Bootstrapping Security]
  If a PPT adversary can convince the honest verifier
  of Algorithm~\ref{alg.bootstrap-verifier}
  holding an admissible tree $\pi$
  of the inclusion of
  a transaction $\tx$, then this transaction will be confirmed by all honest parties
  within $u$ slots, except with negligible probability in the security parameter.
\end{lemma}
\begin{proof}[Sketch]
\end{proof}

By the \emph{chain growth} property, at least such
$k+1$ blocks must exist, and by the \emph{honest subsequence} property, at least one block $b$ in $\chain^*$
must be honestly generated. Since $b$ extends (or is equal to) $\chain^*[0]$, therefore $\chain^*[0]$
contains a correct commitment to the slot leaders of the next epoch, even if it was adversarially generated.
This commitment can be used to obtain the most recent $2k$ slot leaders from the prover, accompanied
by the respective Merkle tree proofs of inclusion.
