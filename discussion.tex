\section{Discussion and Future Work}

\noindent
\textbf{Interactivity.}
The interactivity in our protocol is undesirable. Contrary to linear bootstrapping protocols in which the
bottleneck in practice comes down to the bandwidth needed to download all the data, in our protocol,
because the communication complexity we have achieved is so low,
the practical bottleneck in the protocol comes down to the interactivity.
This interactivity is inherent to our construction and is difficult to remove.
In contrast to standard complexity
proof protocols (such as zero-knowledge proofs), our interactivity cannot be removed using the Fiat--Shamir
heuristic~\cite{fiatshamir}. The reason here is that we require two provers to challenge each other under the watchful
eyes of the verifier. Such behavior cannot be emulated by the random sampling that Fiat--Shamir would offer,
in which a \emph{single} prover and \emph{single} verifier interaction is emulated.
However, even though the protocol has inherent logarithmic interactivity, in practice the actual number
of interaction rounds can be reduced with a couple of techniques:

\begin{enumerate}
\item When the attacker is challenging
the defender, instead of the attacker asking the defender to \emph{open left} or \emph{open right},
the attacker can accompany the request with his own version of what nodes he expects to see on the
\emph{left} and on the \emph{right}~\cite{inside-arbitrum}.
That
way, the defender can immediately be asked to open the next level, wherever it might be unmatching,
too. This reduces the number of
rounds in the interaction by a factor of two. But this technique cannot be applied multiple times
(preemptively opening the tree many subsequent levels deep), because
it quickly leads to an exponential blowup of communication cost with meager savings in interactivity.
\item In case both trees have the same
claimed size, two bisection games are required, but these can run in parallel: When the one party responds to the challenge
of the other, he can accompany the first challenge response with a different challenge of his own. The paths of the
tree that are opened in these parallel games may be completely different, and there is no interaction between
the two games. This optimization has the effect
that, even if the two trees have the same size, the number of interactions will be the same as if
only one bisection game was played.
\item The base of the logarithm can be increased from $2$ to
a larger number. This is a trade-off: the larger the base of the logarithm, the fewer rounds of
interactivity are required, but at the same time, more data need to be sent in each round of interaction.
The base of the logarithm parameter does not have to be preset, and can be specified by the verifier
depending on factors such as the network connection performance, to balance latency (high latency
means we want less interactivity) and bandwidth (low bandwidth means we want more interactivity
but small amounts of data per round of interaction).
\item The tournament
between multiple provers can involve bisection games played in parallel. If the verifier speaks to $8$
provers, once every prover commits to his tree root, the verifier can have the provers compete with
each other in $4$ bisection games running simultaneously over the network, to reduce latency.
The $4$ remaining provers can play against each other in $2$ bisection games running in parallel,
and so forth. Although this does not decrease the number of interactions, the actual network delay
will be significantly reduced as compared to challenging the parties one-by-one.
\end{enumerate}

Despite the above optimizations, the asymptotic interaction remains $\mathcal{O}(\log n)$.
The problem of \emph{Non-Interactive} Proof of Proof-of-Stake (NIPoPoS), PoPoS that can run in
just a single interaction, is left for future work.

\noindent
\textbf{Trusted Setup.}
Our protocol was developed in the standard model and does not require a trusted setup.
The underlying proof-of-stake protocol \emph{does} have some trusted setup assumption:
The genesis epoch contains the public keys of a committee who, at the time of the epoch,
were assumed to have an honest majority. The fact that our protocol does not leverage that
trust to build zero knowledge or other constructions has certain advantages. We only use
hashes and signatures. This makes the protocol easy to understand and straightforward to implement,
with a limited attack surface, a great advantage in security-critical protocols. But also,
in case the trusted setup assumption is violated, our protocol can readily provide evidence
of what exactly happened, because it does not rely on that assumption: At the exact point
where two provers disagree at the bisection game, the evidence presented illustrates two
different epochs at the same index accompanied by their different randomnesses, a situation
that should not occur. In contrast,
a zero-knowledge proof in case of trusted setup failure can lead to decisions in which
foul play may be undetectable, and certainly the exact point and conditions of first
failure are difficult to pinpoint. Additionally, the simplicity of our construction makes
it possible to implement very efficient provers and verifiers in practice. Despite the
slightly worse asymptotics, we expect our construction to perform better in concrete
terms than zero knowledge proof systems that feature large constants in their complexities.

\noindent
\textbf{Proof-of-Work extensions.}
Our tree-based protocol cannot be readily extended to proof-of-work. To see why, consider
a ``chain tree'' similar to our handover tree, but storing the proof-of-work chain in its leaves.
One could na\"ively imagine that two provers can run a similar bisection protocol to
find the first difference in this tree. But note that, contrary to proof-of-stake epochs,
there can be multiple trees that are \emph{admissible} here, and they are not prefixes of one
another. For example, if a mining adversary chooses to mine a secret chain starting
somewhere in the middle of the honest chain, she will end up with a ``chain tree'' whose first
difference with the honest tree will be the first secret block. At that point, the honest
verifier has nothing to see; both trees look equally valid when the leaves are revealed. Therefore,
this scheme cannot be readily applied to \emph{chains}, but must remain constructed on top of
\emph{epochs}, as we chose to do. The closest related proof-of-work construction is FlyClient~\cite{flyclient}
in which leaves are opened at random until foul play is detected, but no bisection games
are played.
