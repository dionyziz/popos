\section{Introduction}

Blockchains are envisioned to be the infrastructure that will handle the currency~\cite{bitcoin} and contract~\cite{buterin,wood}
law of the future. Towards this goal, they must be scalable, long-lived systems that will
survive for decades to come. However, their current design falls short of this goal. One
aspect of its lack of efficiency stems from the fact that the chain grows at a predetermined
rate in expectation~\cite{sok}. The number of blocks therefore increases linearly as time goes by.
When a new node boots up for the first time and it needs to bootstrap only from the genesis
block, it needs to download the whole chain that was ever created throughout history---or
at the very least its block headers. Alternatively, it can rely on a trusted third party,
but this contradicts the most important design principle of decentralized systems.

This \emph{bootstrapping problem} has been successfully resolved in the proof-of-work
setting. The communication needed for proof-of-work bootstrapping is drastically reduced
by transmitting only a
\emph{sample} of blocks. This sample is sufficient to convince
a verifier which chain is the longest. Such systems rely on the
computational difficultly of creating long chains. However, proof-of-work systems are
quickly being abandoned in favour of new proof-of-stake systems. In the proof-of-stake
setting, no such compression is currently possible. The central problem is that, in the
stake setting, an adversary can create chains of arbitrary length without incurring much
computational cost. Besides, the central assumption in stake systems is that the
majority of \emph{stake} at any point in time is honestly controlled, but no limits are
imposed on the computational power of the adversary (beyond standard polynomiality).
The---until now---unsolved problem of superlight proof-of-stake clients is a thorn in the
side of proof-of-stake adoption.

In this work, we put forth the first succinct Proof of Proof-of-Stake (PoPoS) protocol.
The protocol involves a verifier and multiple provers, at least one of which is assumed
to be honest (this is a standard
assumption~\cite{backbone,backbone-new,varbackbone,pass-asynchronous}). The protocol is interactive:
The verifier interacts with the provers in multiple rounds. The verifier initially requests from
each prover a proof about the current state of the chain. After these proofs are received,
the verifier interrogates each prover in a bisection game until any adversarial claims
are ruled out. The number of interactions of the bootstrapping protocol is \emph{logarithmic}
in the execution time of the whole system, and the communication complexity is also \emph{logarithmic}
in total (as the communication in each round is constant). This constitutes an exponential
improvement over previous work.

Our construction has two main applications: \emph{Superlight} proof-of-stake clients that can
bootstrap very efficiently, and \emph{sidechains} that allow the exchange of information between
a source proof-of-stake blockchain and any smart-contract-enabled destination blockchain.

\noindent
\emph{Contributions.} In summary, our contributions are as follows:

\begin{enumerate}
  \item We give the first formal definition for succinct Proof of Proof-of-Stake (PoPoS) protocols.
  \item We put forth a solution to the long-standing problem of efficient Proof-of-Stake bootstrapping.
        Our solution is exponentially better than previous work.
  \item We prove our construction is secure in the standard model on top of Ouroboros proof-of-stake.
  \item In addition to efficient clients, we describe how our constuction can stand at the heart of
        sidechain constructions with a proof-of-stake source chain.
\end{enumerate}

\noindent
\emph{Related work.} The problem of proof-of-work bootstrapping has been explored in the interactive~\cite{popow}
and non-interactive~\cite{nipopows} setting using various constructions from superblocks~\cite{logspace,compactsuperblocks}
to Fiat--Shamir~\cite{fiatshamir} sampling~\cite{flyclient} and proven secure in the Bitcoin Backbone
model~\cite{backbone,backbone-new,varbackbone}. Such constructions can be adopted without
forking~\cite{velvet,velvet-nipopows} and have been deployed in practice~\cite{gas-efficient}. They have
also been used to deploy one-way~\cite{burn} and two-way sidechains~\cite{sidechains,pow-sidechains,crosschain-sok}.

The first provably secure proof-of-stake protocol was introduced in
Ouroboros~\cite{ouroboros} and later extended~\cite{praos,genesis}. Several attempts to improve the efficiency
of clients and sidechains have been proposed~\cite{pos-sidechains,mithril}, but they all achieve only concrete gains in efficiency
and no asymptotic improvement. In the trusted setup model, zkSNARKS have been used to achieve constant communication complexity
in Coda~\cite{coda} and Plumo~\cite{plumo}.
Bisection games like ours have been used in the context of verifiable computation~\cite{refereed-computation}
and in blockchain settings for the efficient execution of smart contracts~\cite{arbitrum}
and for wallet metadata~\cite{wallets}.

